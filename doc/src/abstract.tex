%!TEX root = ../dissertacao.tex
Security issues are one of the biggest internet banking challenges, as such systems are highly targeted by ciber criminals. Distinct security protocols were designed in order provide reliability to online banking transactions. The Dynamic Authorization Protocol is an interesting approach, where a shared key is defined between the User and the Bank for further being used in future online banking transactions. For each transaction, the User is challenged to retrieve a unique transaction code, using a smartphone as a third-party validation entity, scanning a QR-code for integrity checks, for being able to carry operations in a untrusted computer. However, the protocol is not formally checked. In this work, we used the Inductive Method theory for formalizing and verifying the protocol and its properties. An extension to this theory was necessary in order to reason about the smartphones and their interaction with users. Parts of the protocol properties were confirmed, meanwhile some issues with privacy and message clearness were identified.