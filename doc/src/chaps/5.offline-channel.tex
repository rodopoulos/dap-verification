%!TEX root = ../../dissertacao.tex
\label{chap:formal-phones}

As seen in Chapter \ref{chap:dap}, the DAP main strategy for securely authorizing transactions is providing a verification phase between the human User and his smartphone device, where the transaction will be revised and further confirmed by the User. All these operations take place in an offline and out-of-band channel, which is supposed to be inviolable by the Spy. Besides the inviolability of the channel, the DAP specification makes other strong assumptions about the smartphone security model, which will be further discussed later in this chapter.

As stated in Chapter \ref{chap:inductive}, the Inductive Method has proper formalizations for the network channel and regular agents behavior, but we stress that it does not take into account the offline channel and the smartphone peculiarities. In order to verify the protocol, we need to provide a reliable and suitable model for this context. In this chapter, we construct this formal model, considering all the real world details surrounding these entities, adapting the protocol assumptions into a conceivable scenario.





\section{DAP Analysis}
The DAP specification describes the User Smartphone as a key component for the protocol operation and, further, vital for its security. In the following, based on the protocol operational steps, we identify and list the required set of actions for a device rightly fulfill the protocol's goals:

\begin{description}
  \item[Cryptographic Calculus:] the device must be able to securely generate and perform cryptographic keys, hashes, calculations and checks. Such operations are constantly performed by the smartphone on the hash verification phase and decryption of the TAN;

  \item[2D code reading capability:] the ability of scanning and decoding any presented QR code through a physical optical reader is what gives the device means of communication with other devices. Recalling the protocol specification, the Smartphone uses its camera for scanning QR codes along other agents' screens, obtaining the two pieces of the symmetric key $K$ and, further, banking transactions details;

  \item[On-screen data output:] in order to show any received message and further operations to the User, the Smartphone must provide a visual physical channel to display such data;

  \item[Physical input:] at last, the User must be able to accept or decline messages presented on the device screen. Therefore, the smartphone must provide a way for the User to input these actions on the phone.
\end{description}
% TODO: procurar aqui se a especificação pede para que o dispositivo tenha a característica de posse

The above described properties can be embedded on a dedicated hardware, other than smartphones, which could be issued by the bank to the protocol users. Each client would have his own device, already packed with any required encryption key, establishing the proper link between Bank and User. Similar schemes have been already implemented by other online banking systems, for instance, One Time Password (OTP) proprietary tokens are delivered to bank clients, in order to compose two-factor authentication and transaction authorization systems for online banking protocols \cite{Bonneau2012, Claessens2002}.

However, the protocol authors enjoyed the convenience of the smartphone. Not only the device appropriately fits the requirements for the protocol mandatory hardware but also has a strong and known acceptance among users around the world {\color{blue} REFERENCIA}. Therefore, the device is a reasonable choice.
% TODO: consertar parágrafo anterior de acordo com a correção da Cláudia: "a requirement that specified (cite chapter before)".



\subsection{Smartphone Scope}
\label{sec:smartphone-first-analysis}
Smartphones are portable devices that have an operational system, which provides the capacity to perform some actions like a personal computer. They can process data, store information and communicate through a network channel.

Recalling the DAP specification \cite[Ch. 4]{Peotta2012}, the establishment of shared key $K$ acts as a bound between User and Bank. It is securely stored at the Bank and User side, where the latter uses its Smartphone as the entrusted entity for this task. Not only a link between agents is established, but the authentication process is also reinforced, given that only these two entities must know such secret.

Plus, the specification also states that any communication between the smartphone and other devices must be done by visual means: either it outputs data through a screen or obtains information by the smartphone camera.

These strategies provide more security based on two main facts, according to the specification: the long-term keys are never disclosed to the User's computer, which is considered insecure, and any necessary communication between the phone and another device is restricted to a non-interposable and reliable channel \cite{Peotta2012}.

However, it is important to note that the smartphone performs more actions than those required by the protocol. Due to its set of capabilities, the smartphone can act like a personal computer, which in a precise sense makes those devices as unreliable as any other online device. Hence, we have possible attack vectors that should be discussed.



\subsection{DAP Assumptions \textit{vs.} Real World Scenarios}
\label{sec:dap-assumptions}
Before going further, it is important to discuss the environment described in the DAP specification and contrast it with the real world. The protocol specification makes strong assumptions about the User smartphone, stating that it cannot be directly accessed or operated by the Spy.

We argue that such claim is idealistic, since recent studies have shown critical security flaws and attacks in the major mobile phones operating systems \cite{ZhouJiang2012,Han2013}, enabling remote operations by an attacker, for instance, which goes directly against the security requirement dictated by the protocol. 

Moreover, Finally, the situation of smartphone robbery must also be taken in account. Here, the Spy could not only operate the smartphone but, if she is skilled enough, she could also exploit the device, obtaining the same privileges as if the smartphone is compromised. This scenario is effectively contemplated in Inductive Method extended theory for smartcards \cite[Ch. 10]{Bella2007}.

Therefore, in contrast with DAP specification, but in accordance with the formal method for verifying a protocol \cite[Ch. 11]{Bella2007}, we should consider a scenario where the Spy can exploit the User smartphone. Since it is feasible, the theft scenario will be considered in all cases. This strategy will arise two possible formalizations for the protocol and introduce new forms of exploiting the protocol. 

Given the fact that the smartphone can connect to the default network channel, it can be reached by the Spy and further compromised by her. Like an ordinary computer, the smartphone can be exploited in such manner that the attacker can disclose stored data, eavesdrop communication, fake data, and perform other malicious actions through remote control.

At the same time, such attack is only possible when the smartphone is connected to the network channel, i.e., the Internet. Thus, we must properly formalize the possible states of mobile phones connectivity with the cited channel. Also, note that the offline channel remains secure and what is compromised it is only the User device.

Finally, the situation of smartphone robbery must also be taken in account. Here, the Spy could not only operate the smartphone but, if she is skilled enough, she could also exploit the device, obtaining the same privileges as if the smartphone is compromised.



\section{Smartphone Formalization}
As a first step towards its formalization, we properly define smartphones, as an entity linked to one, and only one, agent:

\begin{definition}
  Smartphones are defined as a bijective relation between the agent set and a free type set, denoting the set of available smartphones.
  \begin{center}
    \texttt{Smartphone : agent} $\longrightarrow$ \texttt{smartphone}
  \end{center}
\end{definition}

Note that such formalization bounds the ownership of a smartphone to one User only, establishing a permanent link between both entities. In the DAP specification, such bound is described in terms of the key $K$, shared between the User's smartphone and the Bank, which identifies the User towards the Bank. This entity is defined in a software context, in the smartphone memory, thereby it may be exploited. Furthermore, a bound between User and Smartphone can also be established by hardware manners, using the SIM card, present in any smartphone in order to the device receive a proper phone number.

However, these details are confined to the protocol implementation in the underlying software executing it. In short, the formalization of the smartphone entity cannot fully comprehend how the application will treat the link between a User and a mobile phone, combining software and hardware aspects. Accordingly, we preserve the theoretical ownership aspect, using a conservative approach, stating that our model establishes the link between the two entities and keep it until the end of all trace.



\subsection{Vulnerabilities}
An evaluation of realistic vulnerabilities for smartphones is presented, in order to fully comprehend how such aspects must be formalized in our final model:

\begin{itemize}
  \item \textbf{Theft:} mobile phones are highly susceptible to robbery, considering that its reduced size also increases its risk to loss and stealing. Hence, we need to formalize devices that are robbed from their owners and used by the Spy, including them in a correspondent set \texttt{stolen};

  \item \textbf{Device control:} mobile phones are reduced personal computers, which can be held by users. As noted, they can also be exploited and controlled remotely by the Spy, giving her the ability to request computational actions and access device data. Therefore, compromised smartphones could be added to the set \texttt{badp}, of compromised agents. However, such capacities are limited to the set of computational operations, which does not include physical actions with the smartphone, like pointing its camera to something. As a result, the formalization of a compromised device must not include such actions, but only the ones achievable by computational means.
\end{itemize}

Additionally, once robbed by the Spy, a smartphone may be in possession of a skilled attacker, which now can easily compromise and exploit the device. Even if it is unlikely, this claim is not idealistic {\color{blue} REFERENCIA}. Hence, we state that if a smartphone is robbed, it is automatically compromised, leading to the relation below.
%
$$stolen \subseteq badp$$



\subsection{Keys}
In this model, agents' long-term keys are now stored in the Smartphone and agents do not have direct access to it anymore. Thus, such keys are left out of agents' initial state. However, key formalization is kept, since it belongs to the agent:
%
$$\texttt{shrK} : agent \longrightarrow key$$
%
How such keys are stored does not require an explicit formalization by the Smartphone model, but does require a description of how they are handled by agents and their devices. This approach flexibilize our specification process for smartphones, leaving any specific aspects to the protocol model specification.

Several smartphone's operational systems present to their users the possibility to be operated until a passphrase is inputted, protecting them from any unwanted access and further illegal actions. However, such behavior is not required by the DAP specification and, given the fact that stolen devices are automatically considered compromised, such security scheme is irrelevant. Thus, we do not present a formalization of PIN-required actions on the smartphone.

Finally, we mention the QR code ciphering scheme that happens over some steps of DAP. The protocol standard presents a matrix barcode scheme for displaying information, using four possible encoding modes: numeric, alphanumeric, binary and kanji. That said, we emphasize that no innate encryption happens in such codification and so, there is no need for a key definition here. {\color{blue} Pegar referências sobre QR code, que aparentemente, possui uma especificação proprietaria}



\subsection{Usability}
\label{ssec:usability}
Usability concerns the characterization of smartphone operations, hence modeling this property affects both legal agents and the Spy. Legal agents only perform legal actions, while the Spy can perform both legal and illegal actions. We investigate which actions are interesting to focus, perceiving how the mobile devices communicate with agents.

The set of Smartphone actions is constrained to physical operations, so stolen devices are prone to illegal actions by the Spy. However, as stated in

Legal actions are done by legal users, consequently it is easy to define the legal operation:
%
\begin{center}
  \textit{legalUse}(\texttt{Smartphone} $A$) $\triangleq$ \texttt{Smartphone} $A \notin$ \texttt{stolen}
\end{center}
%
As discussed, the illegal action is constrained to physical actions, thus, for performing any of these, the Spy must physically posses the smartphone, by robbery, so:
%
\begin{center}
  \textit{illegalUse}(\texttt{Smartphone} $A$) $\triangleq$ \texttt{Smartphone} $A \in$ \texttt{stolen}
\end{center}
%
It is important to also discuss the smartphone usability for the Spy, given that she can also perform legal actions. In that sense, we stress that she does not need to use her own card illegally, given that such action would not result in any more relevant information to her knowledge set. Hence:
%
\begin{center}
  \texttt{Smartphone} \textit{Spy} $\notin$ \texttt{stolen}
\end{center}
%
The same is assumed to the server, since it does not use any smartphone.



\subsection{Events}
Having the notions of smartphone threats and usage, we are able to go forward and define the possible event among protocol traces. We extend the \texttt{event} datatype as follows:
%
\begin{equation*}
  \begin{split}
    \texttt{datatype event} \triangleq \
    & \texttt{\textbf{Says} agent agent msg} \\
    & \texttt{\textbf{Notes} agent msg} \\
    & \texttt{\textbf{Gets} agent msg} \\
    & \texttt{\textbf{Inputs} agent smartphone msg} \\
    & \texttt{\textbf{Gets\_s} smartphone msg} \\
    & \texttt{\textbf{Outputs} smartphone agent msg} \\
    & \texttt{\textbf{Gets\_a} agent msg}
  \end{split}
\end{equation*}

The regular network events are kept as in Chapter \ref{chap:inductive}, while four new events are added, which describe the agents and smartphone interactions with the offline channel. We depict each event below:

\begin{description}
  \item[\texttt{Inputs}] defines the act of an agent sending data to a smartphone. Hence, this event describes when a smartphone scans data from some agent's screen;

  \item[\texttt{Gets\_s}] defines the reception event at the smartphone side, in respect to a \textit{Inputs} event;

  \item[\texttt{Outputs}] defines the sending of data from a smartphone to an agent, using the smartphone screen. Like the \textit{Inputs} event, it envolves both entities;

  \item[\texttt{Gets\_a}] finally, defines the reception of data by an agent from the offline channel.
\end{description}

It is easy to note that the new events allow a proper distinction from messages received on the regular channel from the ones received in the new channel. Yet, the Spy still can forge messages on compromised devices and send them through these channels.

The creation of proper events for message reception in offline channels, \texttt{Gets\_s} and \texttt{Gets\_a}, provides a reliable way for smartphones correctly receive messages sent by \texttt{Inputs} events and agents receive messages sent by \texttt{Outputs} events, respectively.

Finally, we need to extend the formal definition of the \textit{used} function, describing how each new event contributes to enrich the set of past messages present in the current trace:

\begin{enumerate}
  \setcounter{enumi}{3}
  \item All components of a message that an agent inputs to his smartphone in a trace are used on that trace:
  \begin{center}
    \texttt{used}((\texttt{Inputs} $A$ $P$ $X$) $\#$ \textit{evs}) $\triangleq$ \texttt{parts} $\{ X\}\ \cup$ \texttt{used} \textit{evs}
  \end{center}

  \item All messages that a smartphone receives as inputs from an agent in a trace do not directly extend those that are used on that trace.
  %
  \begin{center}
    \texttt{used}((\texttt{Gets\_s} $P$ $X$) $\#$ \textit{evs}) $\triangleq$ \texttt{used} \textit{evs}
  \end{center}

  \item All components of a message that an agent's smartphone outputs in a trace are used on that trace:
  %
  \begin{center}
    \texttt{used}((\texttt{Outputs} $A$ $P$ $X$) $\#$ \textit{evs}) $\triangleq$ \texttt{parts} $\{ X\}\ \cup$ \texttt{used} \textit{evs}
  \end{center}

  \item All messages that an agent receives as inputs from a smartphone in a trace do not directly extend those that are used on that trace.
  %
  \begin{center}
    \texttt{used}((\texttt{Gets\_a} $A$ $X$) $\#$ \textit{evs}) $\triangleq$ \texttt{used} \textit{evs}
  \end{center}
\end{enumerate}

\noindent We stress that Cases 5 and 7 do not extend the set of used components, because its corresponding events already include these components in such set.



\subsection{Agents' Knowledge}
\label{ssec:agents-knowledge-smartphone}
With new the events, the agents' knowledge sets are now subject to changes compared to the ones defined in Chapter \ref{chap:inductive}. The smartphone is a device that does not necessarily disclose its data to its agents, but it requires communication in order to enrich the agent's knowledge set. At the same time, we observe that compromised devices will disclose its secrets to the Spy. However, we stress that such situation only happens to smartphones connected to the unreliable channel, that is, the Internet.

The definitions on the agents' initial knowledge set are properly updated, following the guideline of Section \ref{ssec:threat-model} and taking into account the secrets held by the smartphone. We omit asymmetric long-term keys for readability:
%
\begin{enumerate}
  \item The Server's initial knowledge set consists of all long-term secrets, stored both in agents and smartphones:
  %
  \begin{equation*}
    \begin{split}
      \texttt{initState Server} \triangleq
      & (\texttt{Key range shrK}) \ \cup \\
      & (\texttt{Key shrK}\{A.\ Card\ A\})
    \end{split}
  \end{equation*}
  %
  \item The knowledge set for legitimate agents remains the same: their own secrets and all public keys. They do not know their smartphone secrets before any communication;

  \item The Spy knows all public keys and all secrets from compromised agents and smartphones, which includes her own:
  %
  \begin{equation*}
    \begin{split}
      \texttt{initState}\ Spy \triangleq\
      & (\texttt{Key shrK bad}) \ \cup \\
      & (\texttt{Key shrK}\{P.\ Smartphone\ P \in badp\ \wedge \\
      & Smartphone\ P \in connected\})
    \end{split}
  \end{equation*}
\end{enumerate}

The function \textit{knows} will also be extended. The new events impacts both on agents and the Spy knowledge set, where the latter needs careful analysis, as follows:
\begin{enumerate}
  \setcounter{enumi}{3}
  \item An agent knows what she inputs to any smartphone in a trace;
  %
  \begin{center}
    \texttt{knows} $A$ ((\texttt{Inputs} $A'$ $P$ $X$) $\#$ \textit{evs}) $\triangleq$ $\begin{cases}
      \{ X\} \cup \texttt{knows}\ A\ evs & \text{if } A = A', \\
      \texttt{knows}\ A\ evs & \text{otherwise}
    \end{cases}$
  \end{center}

  \item No legal agent can extend his knowledge with any of the message received by any smartphone in a trace, since he already knows given the case of \texttt{Inputs} event. However, if the smartphone is connected to the network and is also compromised, the Spy can learn any information received by it.
  %
  \begin{center}
    \texttt{knows} $A$ ((\texttt{Gets\_s} $P$ $X$) $\#$ \textit{evs}) $\triangleq$ $\begin{cases}
      \{ X\} \cup \texttt{knows}\ A\ evs & \text{if } A = Spy, P \in \textit{connected} \text{ and } P \in badp\\
      \texttt{knows}\ A\ evs & \text{otherwise}
    \end{cases}$
  \end{center}

  \item An agent, including the Spy, knows what her smartphone outputs in a trace, given that reception in the offline channel is guaranteed. Also, the Spy knows what all connected and compromised smartphones output.
  %
  \begin{center}
    \texttt{knows} $A$ ((\texttt{Outputs} $P$ $A$ $X$) $\#$ \textit{evs}) $\triangleq$ $\begin{cases}
      \{ X\} \cup \texttt{knows}\ A\ evs & \text{if } A = A' \text{ or if } A = Spy, P \in \textit{connected} \text{ and } P \in badp \\
      \texttt{knows}\ A\ evs & \text{otherwise}
    \end{cases}$
  \end{center}

  \item An agent, other than the Spy, knows what she receives from his smartphone in a trace.
  %
  \begin{center}
    \texttt{knows} $A$ ((\texttt{Gets\_a} $A$ $X$) $\#$ \textit{evs}) $\triangleq$ $\begin{cases}
      \{ X\} \cup \texttt{knows}\ A\ evs & \text{if } A = A' \text{ and } A \neq Spy \\
      \texttt{knows}\ A\ evs & \text{otherwise}
    \end{cases}$
  \end{center}
\end{enumerate}



\section{Threat Model}
Traditional protocols could have their threat model defined by rule \texttt{Fake} and some other definitions concerning agents knowledge. The introduction of smartphone events and further modifications in knowledge sets induce new kinds of aspects to be considered.

Smartphones can be exploited through illegal actions, as defined in Section~\ref{ssec:usability}. Our model does not allow the Spy to control the offline channel, but she has control over Smartphone inputs and outputs. Therefore, given a illegally usable smartphone, the Spy can send fake messages to it or outputs fake messages to agents from her smartphone, pretending to be the compromised one. Such events should not be mixed with the ability of faking message in the regular network channel, in order to preserve behaviors linked to reception of messages from a specific channel.

Finally, rule \texttt{Fake} is extend in order to embrace such details, producing two more events besides the regular one. All other aspects concerning agents' knowledge sets are already described in Section~\ref{ssec:agents-knowledge-smartphone}.


\section{Protocol Model}
