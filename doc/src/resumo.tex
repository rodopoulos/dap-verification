Questões em segurança compõem grande parte dos desafios das soluções de \textit{internet banking}. Diferentes protocolos de segurança foram projetados visando prover confiabilidade para transações bancárias \textit{online}. O Protocolo de Autorização Dinâmica é uma estratégia interessante, onde uma chave compartilhada entre o banco e o usuário é estabelecida para futuras mensagens. Para cada transação, o usuário é desafiado a recuperar e apresentar uma chave de transação única ao servidor, usando um \textit{smartphone} como uma entidade externa de validação, escaneando um QR-code para testes de integridade. Dessa forma, o usuário consegue realizar operações em um computador inseguro e validá-las em um canal \textit{offline}. Entretanto, o protocolo não é formalmente verificado. Neste trabalho, pretendemos realizar a especificação e verificação formal do protocolo, usando uma técnica de prova de teoremas bem estabelecida, o Método Indutivo. No fim, esperamos ter um dos dois resultados: um certificado formal de correção das crenças de segurança do protocolo ou um contra-exemplo válido para a falha do mesmo.
