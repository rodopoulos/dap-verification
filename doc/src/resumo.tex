%!TEX root = ../dissertacao.tex
Questões em segurança compõem grande parte dos desafios das soluções de \textit{internet banking}. Diferentes protocolos de segurança foram projetados visando prover confiabilidade para transações bancárias \textit{online}. O Protocolo de Autorização Dinâmica segue um esquema onde uma chave compartilhada entre o banco e o usuário é estabelecida para uso em futuras transações bancárias online. Em cada transação, o usuário é desafiado a recuperar e apresentar uma chave de transação única ao servidor, usando um \textit{smartphone} como uma entidade externa de validação, escaneando um QR-code para testes de integridade. Dessa forma o usuário consegue realizar operações em um computador inseguro e validá-las em um canal \textit{offline}. Entretanto, o protocolo não é formalmente verificado. Neste trabalho, utilizamos a teoria do Método Indutivo para formalizar e verificar as propriedades do protocolo. Uma extensão desta teoria foi necessário para modelar o funcionamento de \textit{smartphones} e sua interação com os usuários. Parte das propriedades do protocolo foram validadas, enquanto alguns problemas relativos a privacidade e clareza nas mensagens foram identificados.